\PassOptionsToPackage{unicode=true}{hyperref} % options for packages loaded elsewhere
\PassOptionsToPackage{hyphens}{url}
%
\documentclass[landscape]{article}
\usepackage{lmodern}
\usepackage{amssymb,amsmath}
\usepackage{ifxetex,ifluatex}
\usepackage{fixltx2e} % provides \textsubscript
\ifnum 0\ifxetex 1\fi\ifluatex 1\fi=0 % if pdftex
  \usepackage[T1]{fontenc}
  \usepackage[utf8]{inputenc}
  \usepackage{textcomp} % provides euro and other symbols
\else % if luatex or xelatex
  \usepackage{unicode-math}
  \defaultfontfeatures{Ligatures=TeX,Scale=MatchLowercase}
\fi
% use upquote if available, for straight quotes in verbatim environments
\IfFileExists{upquote.sty}{\usepackage{upquote}}{}
% use microtype if available
\IfFileExists{microtype.sty}{%
\usepackage[]{microtype}
\UseMicrotypeSet[protrusion]{basicmath} % disable protrusion for tt fonts
}{}
\IfFileExists{parskip.sty}{%
\usepackage{parskip}
}{% else
\setlength{\parindent}{0pt}
\setlength{\parskip}{6pt plus 2pt minus 1pt}
}
\usepackage{hyperref}
\hypersetup{
            pdfborder={0 0 0},
            breaklinks=true}
\urlstyle{same}  % don't use monospace font for urls
\usepackage[margin=1in]{geometry}
\usepackage{graphicx,grffile}
\makeatletter
\def\maxwidth{\ifdim\Gin@nat@width>\linewidth\linewidth\else\Gin@nat@width\fi}
\def\maxheight{\ifdim\Gin@nat@height>\textheight\textheight\else\Gin@nat@height\fi}
\makeatother
% Scale images if necessary, so that they will not overflow the page
% margins by default, and it is still possible to overwrite the defaults
% using explicit options in \includegraphics[width, height, ...]{}
\setkeys{Gin}{width=\maxwidth,height=\maxheight,keepaspectratio}
\setlength{\emergencystretch}{3em}  % prevent overfull lines
\providecommand{\tightlist}{%
  \setlength{\itemsep}{0pt}\setlength{\parskip}{0pt}}
\setcounter{secnumdepth}{0}
% Redefines (sub)paragraphs to behave more like sections
\ifx\paragraph\undefined\else
\let\oldparagraph\paragraph
\renewcommand{\paragraph}[1]{\oldparagraph{#1}\mbox{}}
\fi
\ifx\subparagraph\undefined\else
\let\oldsubparagraph\subparagraph
\renewcommand{\subparagraph}[1]{\oldsubparagraph{#1}\mbox{}}
\fi

% set default figure placement to htbp
\makeatletter
\def\fps@figure{htbp}
\makeatother

\usepackage{booktabs}
\usepackage{longtable}
\usepackage{array}
\usepackage{multirow}
\usepackage{wrapfig}
\usepackage{float}
\usepackage{colortbl}
\usepackage{pdflscape}
\usepackage{tabu}
\usepackage{threeparttable}
\usepackage{threeparttablex}
\usepackage[normalem]{ulem}
\usepackage{makecell}
\usepackage{xcolor}

\author{}
\date{\vspace{-2.5em}}

\begin{document}

\hypertarget{lowcountry-ej-results}{%
\section{Lowcountry EJ Results}\label{lowcountry-ej-results}}

\hypertarget{section}{%
\section{--}\label{section}}

\hypertarget{table-1.-conservation-reserves-by-data-source.}{%
\subsubsection{Table 1. Conservation Reserves by Data
Source.}\label{table-1.-conservation-reserves-by-data-source.}}

Square miles of conservation land with number of reserves in
parentheses. All reported numbers are unfiltered from data source with
the exception being multipart polygons clipped to the Lowcountry region.

\includegraphics[width=5.24in]{/var/folders/xg/b293kx9d1fvgkvdm9z_plzr80000gn/T//RtmpGwjw57/filef672636b51b7}

\hypertarget{table-2.-demographic-comparison-by-region.}{%
\subsubsection{Table 2. Demographic Comparison by
Region.}\label{table-2.-demographic-comparison-by-region.}}

Area and Beneficiary Population is total for Lowcountry region, but mean
for private and public beneficiary zones. Population Density is persons
per square mile. Results are for the 10 mile beneficiary zone and 0.2
mile conservation reserve buffer zone pairing.

\includegraphics[width=9.3in]{/var/folders/xg/b293kx9d1fvgkvdm9z_plzr80000gn/T//RtmpGwjw57/filef672349932b2}
\pagebreak

\hypertarget{table-2.1-urban-v.-rural.-demographic-comparison-by-region.}{%
\subsubsection{Table 2.1 (URBAN v. RURAL). Demographic Comparison by
Region.}\label{table-2.1-urban-v.-rural.-demographic-comparison-by-region.}}

Area and Beneficiary Population is total for Lowcountry region, but mean
for private and public beneficiary zones. Population Density is persons
per square mile. Results are for the 10 mile beneficiary zone and 0.2
mile conservation reserve buffer zone pairing.

\includegraphics[width=9in]{/var/folders/xg/b293kx9d1fvgkvdm9z_plzr80000gn/T//RtmpGwjw57/filef672416c3c41}
\pagebreak

\hypertarget{table-2.2-expanded.-demographic-comparison-by-region.}{%
\subsubsection{Table 2.2 (EXPANDED). Demographic Comparison by
Region.}\label{table-2.2-expanded.-demographic-comparison-by-region.}}

Area and Beneficiary Population is total for Lowcountry region, but mean
for private and public beneficiary zones. Population Density is persons
per square mile. Results are from analysis with data filtered by source
(i.e., Private is NCED in GA and SC-TNC in SC; Public is PAD-US for both
states). Numbers shown are for the 10 mile beneficiary zone and 0.2 mile
conservation reserve buffer zone pairing.

\includegraphics[width=9in]{/var/folders/xg/b293kx9d1fvgkvdm9z_plzr80000gn/T//RtmpGwjw57/filef672137c79ce}

\end{document}
